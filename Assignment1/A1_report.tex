%%%%%%%%%%%%%%%%%%%%%%%%%%%%%%%%%%%%%%%%%
% Short Sectioned Assignment
% LaTeX Template
% Version 1.0 (5/5/12)
%
% This template has been downloaded from:
% http://www.LaTeXTemplates.com
%
% Original author:
% Frits Wenneker (http://www.howtotex.com)
%
% License:
% CC BY-NC-SA 3.0 (http://creativecommons.org/licenses/by-nc-sa/3.0/)
%
%%%%%%%%%%%%%%%%%%%%%%%%%%%%%%%%%%%%%%%%%

%----------------------------------------------------------------------------------------
%   PACKAGES AND OTHER DOCUMENT CONFIGURATIONS
%----------------------------------------------------------------------------------------

\documentclass[paper=a4, fontsize=11pt]{scrartcl} % A4 paper and 11pt font size

\usepackage[T1]{fontenc} % Use 8-bit encoding that has 256 glyphs
\usepackage{fourier} % Use the Adobe Utopia font for the document - comment this line to return to the LaTeX default
\usepackage[english]{babel} % English language/hyphenation
\usepackage{amsmath,amsfonts,amsthm} % Math packages

\usepackage{lipsum} % Used for inserting dummy 'Lorem ipsum' text into the template

\usepackage{sectsty} % Allows customizing section commands
\allsectionsfont{\centering \normalfont\scshape} % Make all sections centered, the default font and small caps

\usepackage{graphicx}
\usepackage{float}

\usepackage{fancyhdr} % Custom headers and footers
\pagestyle{fancyplain} % Makes all pages in the document conform to the custom headers and footers
\fancyhead{} % No page header - if you want one, create it in the same way as the footers below
\fancyfoot[L]{} % Empty left footer
\fancyfoot[C]{} % Empty center footer
\fancyfoot[R]{\thepage} % Page numbering for right footer
\renewcommand{\headrulewidth}{0pt} % Remove header underlines
\renewcommand{\footrulewidth}{0pt} % Remove footer underlines
\setlength{\headheight}{13.6pt} % Customize the height of the header

\numberwithin{equation}{section} % Number equations within sections (i.e. 1.1, 1.2, 2.1, 2.2 instead of 1, 2, 3, 4)
\numberwithin{figure}{section} % Number figures within sections (i.e. 1.1, 1.2, 2.1, 2.2 instead of 1, 2, 3, 4)
\numberwithin{table}{section} % Number tables within sections (i.e. 1.1, 1.2, 2.1, 2.2 instead of 1, 2, 3, 4)

\setlength\parindent{0pt} % Removes all indentation from paragraphs - comment this line for an assignment with lots of text


%----------------------------------------------------------------------------------------
%   TITLE SECTION
%----------------------------------------------------------------------------------------

\newcommand{\horrule}[1]{\rule{\linewidth}{#1}} % Create horizontal rule command with 1 argument of height

\title{ 
\normalfont \normalsize 
\textsc{McGill University} \\ [25pt] % Your university, school and/or department name(s)
\horrule{0.5pt} \\[0.4cm] % Thin top horizontal rule
\huge Assignment 1 \\ % The assignment title
\horrule{2pt} \\[0.5cm] % Thick bottom horizontal rule
}

\author{
    Nabil Chowdhury \\
    \small{ID: 260622155} \\
    \small{COMP 551}
} % Your name

\date{\normalsize\today} % Today's date or a custom date

\begin{document}

\maketitle % Print the title

%--------------
%   PROBLEM 1
%--------------

\section{Model Selection}
%--------------
\subsection{Fitting 20-degree polynomial to the dataset}
Dataset-1 consists of a real-valued scalar as input, and a real-valued scalar as output. In order to fit a 20 degree polynomial, we must create features \textit{x\textsuperscript{2}, x\textsuperscript{3},..., x\textsuperscript{20}}, since \textit{x\textsuperscript{1}} is already given to us. We must also include \textit{x\textsuperscript{0} = 1} in the input (the bias term). We must find \textit{w\textsubscript{0}, w\textsubscript{1},..., w\textsubscript{20}} such that \textit{\^{y} = w\textsubscript{0} + w\textsubscript{1}x\textsuperscript{1} + ... + w\textsubscript{20}x\textsuperscript{20}}, where \^{y} is the prediction. \\

Applying the closed form solution  \(w = (X^TX)^{-1}X^Ty\) yields the least-squares weights \textit{w} for the polynomial. Using these weights, the following mean-squared-errors for the training, validation and test sets were calculated:

\begin{center}
\begin{tabular}{ |c|c| } 
    \hline
    \textbf{Set} & \textbf{MSE} \\ 
    \hline
    Train & 7.15259 \\
    Valid & 458.64632 \\ 
    Test & 17.25163 \\
    \hline
\end{tabular}
\end{center}

Here is the fit on the training, validation, and test sets:
\begin{figure}[H]
    \includegraphics[width=\linewidth]{q1p1.png}
    \caption{Fit of 20-degree polynomial without regularization}
    \label{fig:q1p1}
\end{figure}


A closer look at the training plot shows oscillations in the fit (i.e. it is trying very hard to fit the training data). This observation, along with a high validation MSE of 458.64632 indicates that the 20-degree polynomial overfits the dataset. In other words, while the training MSE is low, the curve does not generalize well to the validation set.

\begin{figure}[H]
    \includegraphics[width=\linewidth]{q1p12.png}
    \caption{Fits training set well}
    \label{fig:q1p12}
    \includegraphics[width=\linewidth]{q1p13.png}
    \caption{Fit does not generalize well to validation set}
    \label{fig:q1p13}
\end{figure}

\subsection{Adding L2 regularization to the model}
Adding L2 regularization to our model requires changing the closed form solution by adding a \( \lambda I \) term to \(X^TX\) as follows:
\[ w = (X^TX + \lambda I)^{-1}X^Ty \]

Varying \(\lambda\) from 0 to 1 with increments of .0001 yields the following results for the training and validation sets:

\begin{figure}[H]
    \includegraphics[width=\linewidth]{q1p14.png}
    \caption{\(\lambda\) vs MSE for training and validation sets}
    \label{fig:q1p14}
\end{figure}

The optimum \(\lambda\), based on the lowest MSE of the validation set is 0.0197. Retraining the model with this chosen lambda yields much better MSEs for the validation and test sets:

\begin{center}
\begin{tabular}{ |c|c| } 
    \hline
    \textbf{Set} & \textbf{MSE} \\ 
    \hline
    Valid & 9.13508 \\ 
    Test & 10.73230 \\
    \hline
\end{tabular}
\end{center}

We can also visualize this regularized fit on the test set:

\begin{figure}[H]
    \includegraphics[width=\linewidth]{q1p15.png}
    \caption{Fit of 20-degree polynomial with optimum lambda}
    \label{fig:q1p15}
\end{figure}

\subsection{Estimation of degree of source polynomial}

Observing the regularized fit of the model indicates two sharp bends (at the ends), with approximately 3 slight bends in between. This indicates that a 6\textsuperscript{th} degree polynomial is a good estimate of the source polynomial.

%--------------
%   PROBLEM 2
%--------------

\section{Gradient Descent For Regression}
%--------------
\subsection{Fitting a linear regression model using Stochastic Gradient Descent}

Dataset-2 provides us with a real-valued scalar as input and a real-valued scalar as output. Using Stochastic Gradient Descent (SGD) with a step size \(\alpha\) of \(10^{-6}\) yields the following learning curve:

\begin{figure}[H]
    \includegraphics[width=\linewidth]{q2p1.png}
    \caption{Epoch vs MSE at \(\alpha=10^{-6}\)}
    \label{fig:q2p1}
\end{figure}

\subsection{Finding the best step size}
The following step sizes were tested to determine which step size yielded the best results for the validation set:  \(10^{-6}, 2\times10^{-6}\, 10^{-5}, 2\times10^{-5}\, 10^{-4}, 2\times10^{-4}\, 10^{-3}, 2\times10^{-3}\, 10^{-2}, 2\times10^{-2}\). Based on these values, the best \(\alpha\) was found to be \(10^{-5}\). 

\begin{figure}[H]
    \includegraphics[width=\linewidth]{q2p2.png}
    \caption{\(\alpha\) vs MSE for the Validation set}
    \label{fig:q2p2}
\end{figure}

\subsection{Evolution of the regression fit during training}
The following figures show the progression of SGD at 0, 2000, 4000, 6000 and 8000 epochs.

\begin{figure}[H]
    \includegraphics[width=\linewidth]{q2p31.png}
    \caption{Fit with randomly initialized weights}
    \label{fig:q2p31}
    \includegraphics[width=\linewidth]{q2p32.png}
    \caption{SGD makes large jump towards optimum weights}
    \label{fig:q2p32}
\end{figure}
\begin{figure}[H]
    \includegraphics[width=\linewidth]{q2p33.png}
    \caption{SGD starts taking smaller steps towards optimum}
    \label{fig:q2p33}
    \includegraphics[width=\linewidth]{q2p34.png}
    \caption{At 6000 epochs}
    \label{fig:q2p34}
\end{figure}
\begin{figure}[H]
    \includegraphics[width=\linewidth]{q2p35.png}
    \caption{Near the end of training}
    \label{fig:q2p35}
    \includegraphics[width=\linewidth]{q2p36.png}
    \caption{End of training}
    \label{fig:q2p36}
\end{figure}

\end{document}
